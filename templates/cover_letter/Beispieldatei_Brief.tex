% File: lettertest.tex, Package: ubletter

% Dieses File ist ein Beispiel für ein LaTeX-Brief mit dem neuen Design der Universität Bern
% Hier ist die Adresse, der Betreff, die Anrede, der Brieftext und die Grussformel einzugeben.

% Mögliche Optionen ubletter: Sprache(german/english...), Absender(personal/institute), Farbe des Logos (color/bw)
\documentclass[personal,german,color]{ubletter}


\begin{document}

\begin{letter}{Birkhäuser Verlag \\ Herr Pierre Schrader \\
               Viaduktstrasse 42\\ 4051 Basel \\ Schweiz}						% Empfängeradresse

\subject{Briefpapier}											% Betreff
\opening{Sehr geehrter Herr Schrader}							% Anrede

% --- Beginn Text ---

Im Begriff des Sprachspiels will Wittgenstein die seiner Meinung
nach falsche Trennung von Sprache und Handeln überwinden. Sprache
ist für ihn ein situativer Funktionszusammenhang, er existiert genau
darin und dadurch, dass Menschen handeln. Sprache ist Praxis.
Sprache an sich existiert nicht. Verständlich werden Wörter nach
Wittgenstein nicht allein aus der gesprochenen oder geschriebenen
Sprachstruktur, sondern erst aus dem Ensemble von Situation,
Handlungs- und Sprechabsicht, den außersprachlichen Äußerungen wie
Gestik und Mimik sowie dem situativen Handeln.

Sich in einer bestimmten Situation angemessen und verständlich
äußern zu können heißt, die Regeln eines Sprachspiels anwenden zu
können. In der Teilnahme versteht und lernt man die Anwendungsregeln
für die Bildung sinnvoller Sätze in gelingender Kommunikation. Die
kommunikationstheoretisch verwendete Metapher Sprachspiel zielt auf
die Komplexität realer Verwendungsformen von Sprache im Gefüge von
Situationen mit wahrhaftigen, d.h. in ihrer ganzen Persönlichkeit
anwesenden Sprechern. Wittgenstein vergleicht die Sprache in diesem
Zusammenhang auch mit einem Schachspiel und einzelne Wörter mit
Schachfiguren. Ein Spiel wie das Schachspiel ist in seinen Regeln
zwar klar definiert, das Zusammenspiel der Figuren bietet aber
unendliche Möglichkeiten der Anwendung dieser Regeln.

Sich in einer bestimmten Situation angemessen und verständlich
äußern zu können heißt, die Regeln eines Sprachspiels anwenden zu
können. In der Teilnahme versteht und lernt man die Anwendungsregeln
für die Bildung sinnvoller Sätze in gelingender Kommunikation. Die
kommunikationstheoretisch verwendete Metapher Sprachspiel zielt auf
die Komplexität realer Verwendungsformen von Sprache im Gefüge von
Situationen mit wahrhaftigen, d.h. in ihrer ganzen Persönlichkeit
anwesenden Sprechern. Wittgenstein vergleicht die Sprache in diesem
Zusammenhang auch mit einem Schachspiel und einzelne Wörter mit
Schachfiguren. Ein Spiel wie das Schachspiel ist in seinen Regeln
zwar klar definiert, das Zusammenspiel der Figuren bietet aber
unendliche Möglichkeiten der Anwendung dieser Regeln.

Sich in einer bestimmten Situation angemessen und verständlich
äußern zu können heißt, die Regeln eines Sprachspiels anwenden zu
können. In der Teilnahme versteht und lernt man die Anwendungsregeln
für die Bildung sinnvoller Sätze in gelingender Kommunikation. Die
kommunikationstheoretisch verwendete Metapher Sprachspiel zielt auf
die Komplexität realer Verwendungsformen von Sprache im Gefüge von
Situationen mit wahrhaftigen, d.h. in ihrer ganzen Persönlichkeit
anwesenden Sprechern. Wittgenstein vergleicht die Sprache in diesem
Zusammenhang auch mit einem Schachspiel und einzelne Wörter mit
Schachfiguren. Ein Spiel wie das Schachspiel ist in seinen Regeln
zwar klar definiert, das Zusammenspiel der Figuren bietet aber
unendliche Möglichkeiten der Anwendung dieser Regeln.

% --- Ende Text ---

\closing{Mit freundlichem Gruss}									% Grussformel
\end{letter}

\end{document}
